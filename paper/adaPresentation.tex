\documentclass{beamer}
\usetheme{Madrid}
\usefonttheme{serif}

% Here are the packages I use for equations, code, inserting images, making graphs, and dark mode document, basically.

\usepackage{amsmath}
\usepackage{listings}
\lstset{
    basicstyle=\small\ttfamily,
    breaklines=true
}
\usepackage{graphicx}

\AtBeginSection[]
{
  \begin{frame}
    \frametitle{Table of Contents}
    \tableofcontents[currentsection]
  \end{frame}
}

\title{Analysis of Block Matching Algorithms for Image Transformation}
\subtitle{ADA Final Project: Questions 5 - 10}
\author{Alejandro Goicochea\and Diego Linares\and Ariana Villegas}
\institute{\inst{}Universidad de Ingeniería y Tecnología}
\date{July 23, 2020}

\begin{document}
  \frame{\titlepage}
  \begin{frame}
    \frametitle{Table of Contents}
    \tableofcontents
  \end{frame}
  \section{Recap of Block Matching}
    \begin{frame}
      \frametitle{Previous Presentation}
      In the previous presentation we designed \textbf{two algorithms} for block matching:
      \begin{itemize}
        \item Greedy / Naive Algorithm
        \item Memoized Algorithm (now improved to DP version).
      \end{itemize}
      \medskip
      For this last part we are including a third algorithm, a DP with \textbf{better weight}.\\
      \medskip
      We are gonna see how each of this performs for image transformation. 
    \end{frame}
  \section{Decoding Images through Luma}
    \subsection{What is Luma?}
      \begin{frame}
        \frametitle{What is Luma?}
        Luma represents the relative luminance of an image. It is often used in video engineering.
        It is calculated with various coefficients, some of which are:
        \begin{itemize}
          \item CCIR 601
          \item BT. 709
          \item SMPTE 240M
        \end{itemize}
        \medskip
        These are the ones used in our implementation.\\
      \end{frame}
    \subsection{Transforming into a Matrix of bits}
      \begin{frame}
        \frametitle{Transforming into a Matrix of bits}
        To transform our images into matrices of bits, we will use the result of calculating the luma of 
        each pixel. What determines if the bit goes to 0 or 1 is another parameter passed to our function.
        \medskip
        We repeat this for every pixel of both images, if the luma is equal or greater than our value, 
        the value is set to 1. Otherwise it is set to 0.
      \end{frame}
  \section{Our Approach}
    \subsection{Base Process for Transformation}
    \subsection{Different Algorithms}
  \section{Examples}
  \section{Conclusion}
    \subsection{Algorithm Effectiveness}
    \subsection{Personal Valoration}

\end{document}