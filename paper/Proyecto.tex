\documentclass[12pt]{article}
\usepackage[margin=1in]{geometry} 
\usepackage{amsmath,amsthm,amssymb,amsfonts}
\usepackage[spanish]{babel}
\usepackage[utf8]{inputenc}
\usepackage{mathtools}
\usepackage{hyperref}
\selectlanguage{spanish}
\newcommand{\N}{\mathbb{N}}
\newcommand{\Z}{\mathbb{Z}}
\usepackage{color}
\usepackage{graphicx}
\usepackage[noend]{algorithmic}

\newenvironment{solution}
  {\renewcommand\qedsymbol{$\blacksquare$}\begin{proof}[Solución]}
  {\end{proof}}

\newcommand{\TITLE}[1]{\item[#1]}

\renewcommand{\algorithmiccomment}[1]{$/\!/$ \parbox[t]{4.5cm}{\raggedright #1}}
% ugly hack for for/while
\newbox\fixbox
\renewcommand{\algorithmicdo}{\setbox\fixbox\hbox{\ {} }\hskip-\wd\fixbox}
% end of hack
%imitando para if
\renewcommand{\algorithmicthen}{\setbox\fixbox\hbox{\ {} }\hskip-\wd\fixbox}
\newcommand{\algcost}[2]{\strut\hfill\makebox[1.5cm][l]{#1}\makebox[4cm][l]{#2}}

%piso techo 
\DeclarePairedDelimiter\ceil{\lceil}{\rceil}
\DeclarePairedDelimiter\floor{\lfloor}{\rfloor}
 
 \newenvironment{ejercicio}[2][Ejercicio]{\begin{trivlist}
\item[\hskip \labelsep {\bfseries #1}\hskip \labelsep {\bfseries #2.}]}{\end{trivlist}}

\newenvironment{problem}[2][Problem]{\begin{trivlist}
\item[\hskip \labelsep {\bfseries #1}\hskip \labelsep {\bfseries #2.}]}{\end{trivlist}}


\title{Proyecto de ADA - Primer Avance}
\author{Alejandro Goicochea \and Diego Linares \and Ariana Villegas}
\date{16 de Junio del 2020}

\begin{document}
\maketitle

\section*{Secuencias}

\subsection*{Pregunta 1 (Voraz)}

\subsubsection*{Algoritmo: Get-Blocks}

\noindent Recibe: Un arreglos A de ceros y unos.

\noindent Devuelve: Un arreglo X de bloques de unos. 

\begin{algorithmic}[1]
  \TITLE{\textsc{Get-Blocks}$(A,p)$}
    \algcost{\textit{cost}}{\textit{times}}
  \STATE i = 1
    \algcost{$c_1$}{$1$}
  \STATE j = 1
    \algcost{$c_1$}{$1$}
  \WHILE{$i\leq p$
    \algcost{$c_3$}{$1$}}
  \STATE tmp = 0
    \algcost{$c_1$}{$1$}
    \WHILE{$A[i]\neq 1$
      \algcost{$c_4$}{$1$}}
    \STATE i += 1
      \algcost{$c_5$}{$0$}
    \ENDWHILE
    \WHILE{$A[i]\neq 0$
      \algcost{$c_4$}{$p+1$}}
    \STATE i += 1
      \algcost{$c_5$}{$p$}
    \STATE tmp += 1
      \algcost{$c_5$}{$p$}
    \ENDWHILE
  	\IF{tmp $\neq 0$ \algcost{$c_6$}{$1$}} 
    \STATE X[j] = tmp
      \algcost{$c_7$}{$1$}
    \STATE j += 1
      \algcost{$c_5$}{$1$}
    \ENDIF 
  \ENDWHILE
  \STATE return X
    \algcost{$c_8$}{$1$}
\end{algorithmic} 

\subsubsection*{Tiempo de ejecución: Get-Blocks}

\noindent Para T(p) el tiempo de ejecución con un array de tamaño p como entrada, tenemos que
\begin{eqnarray*}
T(p)&=&c_1+c_1+c_3+c_1+c_4+c_4\cdot (p+1)+c_5\cdot p+c_5\cdot p+c_6+c_7+c_5+c_8 \\
T(p)&=&c_4\cdot p+2c_5\cdot p+3c_1+c_3+2c_4+c_5+c_6+c_7+c_8 
\end{eqnarray*}

\noindent Si $a=c_4+2c_5$ y $b=3c_1+c_3+2c_4+c_5+c_6+c_7+c_8$, entonces
$$T(p)=ap+b$$

\noindent \textbf{Nota: } Para el análisis de los algoritmos propuestos en los siguientes apartados no se tomará en cuenta el tiempo de generar los bloques en los array de entrada.

\subsubsection*{Algoritmo: Min-Matching-Greedy}

\noindent Recibe: Dos arreglos A y B de ceros y unos de tamaño p, con n bloques y m bloques respectivamente (los valores de n y m no son recibidos como entrada). 

\noindent Devuelve: Un matching entre A y B, no necesariamente óptimo, y su peso.

\begin{algorithmic}[1]
  \TITLE{\textsc{Min-Matching-Greedy}$(A,B)$}
    \algcost{\textit{cost}}{\textit{times}}
  \STATE X = Get-Blocks(A,p)
    \algcost{$.$}{$.$}
  \STATE Y = Get-Blocks(B,p)
    \algcost{$.$}{$.$}
  \STATE n = X.size
    \algcost{$c_1$}{$1$}
  \STATE m = Y.size
    \algcost{$c_2$}{$1$}  
  \STATE Match = $\emptyset$
    \algcost{$c_3$}{$1$}
  \STATE $w = 0$
    \algcost{$c_4$}{$1$}
  \FOR{$i=1$ TO $min(n,m)-1$
    \algcost{$c_5$}{$m$}}  
  \STATE Match = Match $\cup \ \{\{i,i\}\}$
    \algcost{$c_6$}{$m-1$}
  \STATE $w$ += X[i]/Y[i]
    \algcost{$c_7$}{$m-1$}
  \ENDFOR  
  \IF{$n>m$ \algcost{$c_8$}{$1$}} 
    \FOR{$i=0$ TO $n-m$
    \algcost{$c_5$}{$n-m+1$}}  
    \STATE Match = Match $\cup \ \{\{m+i,m\}\}$
      \algcost{$c_6$}{$n-m$}
    \STATE $w$ += X[m+i]/Y[m]
      \algcost{$c_7$}{$n-m$}
    \ENDFOR 
  \ELSE
    \FOR{$i=0$ TO $n-m$
    \algcost{$c_5$}{$0$}}  
    \STATE Match = Match $\cup \ \{\{n,n+i\}\}$
      \algcost{$c_6$}{$0$}
    \STATE $w$ += X[n]/Y[n+i]
      \algcost{$c_7$}{$0$}
    \ENDFOR 
  \ENDIF 
  
  \STATE return $w$, Match
    \algcost{$c_9$}{$1$}
\end{algorithmic}

\subsubsection*{Tiempo de ejecución: Min-Matching-Greedy}

\noindent Para T(n,m) el tiempo de ejecución con un array de tamaño p como entrada, tenemos que
\begin{eqnarray*}
T(n,m)&=&c_1+c_2+c_3+c_4+c_5\cdot m+c_6\cdot (m-1)+c_7\cdot (m-1)+c_8+c_5\cdot (n-m+1)+\\
& &c_6\cdot (n-m)+c_7\cdot (n-m)+c_9 \\
T(n,m)&=&c_5\cdot m+c_6\cdot m+c_7\cdot m-c_5\cdot m-c_6\cdot m-c_7\cdot m+c_5\cdot n+c_6\cdot n+c_7\cdot n+c_1 \\
& &+c_2+c_3+c_5-c_6-c_7+c_8+c_9 
\end{eqnarray*}

\noindent Si $a=c_5+c_6+c_7-c_5-c_6-c_7=0$, $b=c_5+c_6+c_7$ y $c=c_1+c_2+c_3+c_5-c_6-c_7+c_8+c_9$, entonces si $n>m$
$$T(n,m)=bn+c$$
Y si $m>n$, tenemos que
$$T(n,m)=am+c$$
Entonces, podemos generalizarlo como
$$T(n,m)=max(am,bn)+c$$


\noindent \textbf{Prueba que $T(n) = O(max(n,m))$} \\

\noindent Note que para $n_0\geq max(a,b)+1$, tenemos que 
\begin{eqnarray*}
max(am,bn)+c &\leq& max(max(a,b)\cdot m, max(a,b)\cdot n)+c \\
&\leq& max(a,b)\cdot max(m,n)+c \\
&\leq& (max(a,b)+1)\cdot max(m,n) \\
&\leq& n_0\cdot max(m,n)
\end{eqnarray*}

\noindent Y como la función $max$ es creciente, tenemos que
\begin{eqnarray*}
0\leq max(am,bn)\leq n_0\cdot max(m,n)
\end{eqnarray*}
\noindent Entonces por definición, concluimos que $T(n) = O(max(n,m))$

\subsection*{Pregunta 2 (Recurrencia)}

\noindent Asumimos que:\\
$X = \textsc{Get-Blocks}(A,p)$ \\
$Y = \textsc{Get-Blocks}(B,p)$ 

\[
OPT(i,j)=\left\{
            \begin{array}{ll}
              \frac{X_i}{Y_j} \ \ \ \ \ \ \ \ \ \ \ \ \ \ \ \ \ \ \ \ \ \ \ \ \ \ \ \ \ \ \ \ \ \ \ \ \ \ \ \ \ \ \ \ \ \ \ \ \ \ \ \ \ \ \ \ \ \ \ \ \ \ \ \ \ \ \ \ \ \ \ \ \ \ i=1 \ \ \wedge \ \ j=1 \\ \\
              \frac{X_i}{\sum_{p=1}^{j}Y_p} \ \ \ \ \ \ \ \ \ \ \ \ \ \ \ \ \ \ \ \ \ \ \ \ \ \ \ \ \ \ \ \ \ \ \ \ \ \ \ \ \ \ \ \ \ \ \ \ \ \ \ \ \ \ \ \ \ \ \ \ \ \ \ \ \ \ \ \ i=1 \ \ \wedge \ \ j>1 \\ \\
              \frac{\sum_{p=1}^{i}X_p}{Y_j} \ \ \ \ \ \ \ \ \ \ \ \ \ \ \ \ \ \ \ \ \ \ \ \ \ \ \ \ \ \ \ \ \ \ \ \ \ \ \ \ \ \ \ \ \ \ \ \ \ \ \ \ \ \ \ \ \ \ \ \ \ \ \ \ \ \ \ i>1 \ \ \wedge \ \ j=1 \\ \\
              min(M_g(i,j), M_d(i,j)) \ \ \ \ \ \ \ \ \ \ \ \ \ \  \ \ \ \ \ \ \ \ \ \ \ \ \ \ \ \ \ \ \ \ \ \ \ \ \ \ \ \ \ \ \ \ \ i>1 \ \ \wedge \ \ j>1 \\
            \end{array}
          \right.
\]

\noindent
$M_g(i,j) = min_{k=1}^{i-1}\{\frac{\sum_{p=k+1}^{i}X_p}{Y_j} + OPT(k,j-1)\}$ \\ \\
$M_d(i,j) = min_{k=1}^{j-1}\{\frac{X_i}{\sum_{p=k+1}^{j}Y_p} + OPT(i-1,k)\}$

\subsection*{Pregunta 3 (Recursivo)}

\noindent Asumimos que:\\
$X = \textsc{Get-Blocks}(A,p)$ \\
$Y = \textsc{Get-Blocks}(B,p)$ 

\subsubsection*{Algoritmo: Group}

\begin{algorithmic}[1]
  \TITLE{\textsc{Group}$(X,Y,i,j)$}
    \algcost{\textit{cost}}{\textit{times}}
  \STATE Min = $\infty$
    \algcost{$c_1$}{$1$}
  \FOR{$k=1$ TO $i-1$ \algcost{$c_2$}{$i$}}  
    \STATE accum = 0
      \algcost{$c_3$}{$i-1$}
    \FOR{$p=k+1$ TO $i$ \algcost{$c_4$}{$\frac{(i-2)(i-1)}{2}$}}  
      \STATE accum += $X[p]$
        \algcost{$c_5$}{$\frac{(i-2)(i-1)}{2}$}
    \ENDFOR  
    \STATE accum /= $Y[j]$
      \algcost{$c_6$}{$i-1$}
    \STATE match, pmin = \textsc{Min-Matching-Recursive}$(X,Y,k,j-1)$ \\
      \algcost{$T_m(i-1,j-1)$}{$\ \ \ \ \ \ \ \ \ \ \ \ 2$}
    \IF{Min$>$accum+pmin \algcost{$c_7$}{$i-1$}} 
      \STATE Min = accum + pmin
        \algcost{$c_8$}{$i-1$}
      \STATE Match = match
        \algcost{$c_9$}{$i-1$}
    \ENDIF 
  \ENDFOR  
  \RETURN Match$\ \cup \ \{[i,j]\},$ Min
    \algcost{$c_1$}{$1$}
\end{algorithmic}

\subsubsection*{Tiempo de ejecución: Group}

\noindent Para $T_g(i,j)$ el tiempo de ejecución con un array de tamaño p como entrada, tenemos que
\begin{eqnarray*}
T_g(i,j)&=&c_1+c_2+c_3\cdot (i-1)+c_4\cdot \frac{(i-2)(i-1)}{2}+c_5\cdot \frac{(i-2)(i-1)}{2}+c_6\cdot (i-1)+\\
& &2\cdot T_m(i-1,j-1)+c_7\cdot (i-1)+c_8\cdot (i-1)+c_9\cdot (i-1)+c_1\\
T_g(i,j)&=&2\cdot T_m(i-1,j-1)+\frac{c_4}{2}\cdot i^2+\frac{c_5}{2}\cdot i^2-3c_4\cdot i-3c_5\cdot i+c_6\cdot i+c_7\cdot i+ \\
& &c_8\cdot i+c_9\cdot i+2c_1+c_2-c_3+3c_4+3c_5-c_6-c_7-c_8-c_9 
\end{eqnarray*}

\noindent Si $a=\frac{c_4}{2}+\frac{c_5}{2}$, $b=-3c_4-3c_5+c_6+c_7+c_8+c_9$ y $c=2c_1+c_2-c_3+3c_4+3c_5-c_6-c_7-c_8-c_9$, entonces
$$T_g(i,j)=2\cdot T_m(i-1,j-1)+ai^2+bi+c$$


\subsubsection*{Algoritmo: Division}

\begin{algorithmic}[1]
  \TITLE{\textsc{Division}$(X,Y,i,j)$}
    \algcost{\textit{cost}}{\textit{times}}
  \STATE Min = $\infty$
    \algcost{$c_1$}{$1$}
  \FOR{$k=1$ TO $j-1$ \algcost{$c_2$}{$j$}}  
    \STATE accum = 0
      \algcost{$c_3$}{$j-1$}
    \FOR{$p=k+1$ TO $j$ \algcost{$c_4$}{$\frac{(j-2)(j-1)}{2}$}}  
      \STATE accum += $Y[p]$
        \algcost{$c_5$}{$\frac{(j-2)(j-1)}{2}$}
    \ENDFOR  
    \STATE accum = $X[i]/$accum
      \algcost{$c_6$}{$j-1$}
    \STATE match, pmin = \textsc{Min-Matching-Recursive}$(X,Y,i-1,k)$ \\
      \algcost{$T_m(i-1,j-1)$}{$\ \ \ \ \ \ \ \ \ \ \ \ 2$}
    \IF{Min$>$accum+pmin \algcost{$c_7$}{$j-1$}} 
      \STATE Min = accum + pmin
        \algcost{$c_8$}{$j-1$}
      \STATE Match = match
        \algcost{$c_9$}{$j-1$}
    \ENDIF 
  \ENDFOR
  \RETURN Match$\ \cup \ \{[i,j]\},$ Min
    \algcost{$c_1$}{$1$}
\end{algorithmic}

\subsubsection*{Tiempo de ejecución: Division}

\noindent Para $T_d(i,j)$ el tiempo de ejecución con un array de tamaño p como entrada, tenemos que
\begin{eqnarray*}
T_d(i,j)&=&c_1+c_2+c_3\cdot (j-1)+c_4\cdot \frac{(j-2)(j-1)}{2}+c_5\cdot \frac{(j-2)(j-1)}{2}+c_6\cdot (j-1)+\\
& &2\cdot T_m(i-1,j-1)+c_7\cdot (j-1)+c_8\cdot (j-1)+c_9\cdot (j-1)+c_1\\
T_d(i,j)&=&2\cdot T_m(i-1,j-1)+\frac{c_4}{2}\cdot j^2+\frac{c_5}{2}\cdot j^2-3c_4\cdot j-3c_5\cdot j+c_6\cdot j+c_7\cdot j+ \\
& &c_8\cdot j+c_9\cdot j+2c_1+c_2-c_3+3c_4+3c_5-c_6-c_7-c_8-c_9 
\end{eqnarray*}

\noindent Si $a=\frac{c_4}{2}+\frac{c_5}{2}$, $b=-3c_4-3c_5+c_6+c_7+c_8+c_9$ y $c=2c_1+c_2-c_3+3c_4+3c_5-c_6-c_7-c_8-c_9$, entonces
$$T_d(i,j)=2\cdot T_m(i-1,j-1)+aj^2+bj+c$$

\subsubsection*{Algoritmo: Min-Matching-Recursive}

\begin{algorithmic}[1]
  \TITLE{\textsc{Min-Matching-Recursive}$(X,Y,i,j)$}
    \algcost{\textit{cost}}{\textit{times}}
  \IF{$i==1$ \AND $j==1$ \algcost{$c_1$}{$1$}} 
    \RETURN $\{[i,j]\}, \ \frac{X[i]}{Y[i]}$
      \algcost{$c_2$}{$1$}
  \ENDIF 
  \IF{$i==1$ \AND $j>1$ \algcost{$c_3$}{$1$}} 
    \STATE match = \{\}
      \algcost{$c_4$}{$1$}
    \STATE accumY = 0
      \algcost{$c_5$}{$1$}
  	\FOR{$p=1$ TO $j$ \algcost{$c_6$}{$j+1$}}  
      \STATE accumY += Y[p]
        \algcost{$c_7$}{$j$}
      \STATE Match = Match$\ \cup \ \{[i,p]\},$ 
        \algcost{$c_8$}{$j$}
    \ENDFOR 
    \RETURN Match$, \ \frac{X[i]}{accumY}$
      \algcost{$c_2$}{$1$}
  \ENDIF 
  \IF{$i>1$ \AND $j==1$ \algcost{$c_3$}{$1$}} 
    \STATE match = \{\}
      \algcost{$c_4$}{$1$}
    \STATE accumX = 0
      \algcost{$c_5$}{$1$}
  	\FOR{$p=1$ TO $i$ \algcost{$c_6$}{$i+1$}}  
      \STATE accumX += X[p]
        \algcost{$c_7$}{$i$}
      \STATE Match = Match$\ \cup \ \{[p,j]\},$ 
        \algcost{$c_8$}{$i$}
    \ENDFOR 
    \RETURN $Match, \ \frac{accumX}{Y[j]}$
      \algcost{$c_2$}{$1$}
  \ENDIF 
  \STATE MatchG, MinG = \textsc{Group}$(X,Y,i,j)$
    \algcost{$T_g(i,j)$}{$1$}
  \STATE MatchD, MinD = \textsc{Division}$(X,Y,i,j)$
    \algcost{$T_d(i,j)$}{$1$}  
  \IF{MinG$>$MinD \algcost{$c_3$}{$1$}} 
    \RETURN MatchD, MinD
      \algcost{$c_2$}{$1$}
  \ENDIF  
  \RETURN MatchG, MinG
    \algcost{$c_2$}{$1$}

\end{algorithmic}

\subsubsection*{Tiempo de ejecución: Min-Matching-Recursive}

\[
T_m(i,j)=\left\{
            \begin{array}{ll}
              1 \ \ \ \ \ \ \ \ \ \ \ \ \ \ \ \ \ \ \ \ \ \ \ \ \ \ \ \ \ \ \ \ \ \ \ \ \ \ \ \ \ \ \ \ \ \ \ \ \ \ \ \ \ \ \ \ \ \ \ \ \ \ \ \ \ \ \ \ \ \ \ \ \ \ i=1 \ \ \wedge \ \ j=1 \\ \\
              j \ \ \ \ \ \ \ \ \ \ \ \ \ \ \ \ \ \ \ \ \ \ \ \ \ \ \ \ \ \ \ \ \ \ \ \ \ \ \ \ \ \ \ \ \ \ \ \ \ \ \ \ \ \ \ \ \ \ \ \ \ \ \ \ \ \ \ \ \ \ \ \ \ \ i=1 \ \ \wedge \ \ j>1 \\ \\
              i \ \ \ \ \ \ \ \ \ \ \ \ \ \ \ \ \ \ \ \ \ \ \ \ \ \ \ \ \ \ \ \ \ \ \ \ \ \ \ \ \ \ \ \ \ \ \ \ \ \ \ \ \ \ \ \ \ \ \ \ \ \ \ \ \ \ \ \ \ \ \ \ \ \ \ i>1 \ \ \wedge \ \ j=1 \\ \\
              4T_m(i-1,j-1)+i^2+i+j^2+j \ \ \ \ \ \ \ \ \ \ \ \ \ \ \ \ \ \ \ \ \ \ \ \ \ \ \ \ \ \ \ \ i>1 \ \ \wedge \ \ j>1 \\
            \end{array}
          \right.
\]

\subsubsection*{Resolver la recurrencia: Min-Matching-Recursive}
\noindent El análisis será para el caso en el que $i=j$ debido a que es el peor caso. Entonces, tenemos que
\begin{eqnarray*}
T_m(i,i) &=& 4T_m(i-1,i-1) + i(i+1) + i(i+1) \\
&=& 4T_m(i-1,i-1) + 2i(i+1) \\
&=& 4(4T_m(i-2,i-2) + 2(i-1)i) + 2i(i+1) \\
&=& 4^2T_m(i-2,i-2) + 4\cdot 2(i-1)i + 2i(i+1) \\
&=& 4^2(4T_m(i-3,i-3) + 2(i-2)(i-1)) + 4\cdot 2(i-1)i + 2i(i+1) \\
&=& 4^3T_m(i-3,i-3) + 4^2\cdot 2(i-2)(i-1) + 4\cdot 2(i-1)i + 2i(i+1) \\
&=& 4^3T_m(i-3,i-3) + 2\sum_{k=0}^{2} 4^k\cdot (i-k-1)(i-k) \\
&=& 4^lT_m(i-l,i-l) + 2\sum_{k=0}^{l-1} 4^k\cdot (i-k-1)(i-k) \\
&=& 4^{i-1}T_m(i-(i-1),i-(i-1)) + 2\sum_{k=0}^{i-2} 4^k\cdot (i-k-1)(i-k) \\
&=& 4^{i-1} + 2\sum_{k=0}^{i-2} 4^k\cdot (i^2+k^2-2ki-i+k) \\
\end{eqnarray*}

\noindent Ahora suponemos que $i=2^p$
\begin{eqnarray*}
T_m(i,i) &=& 4^{i-1} + 2\sum_{k=0}^{i-2} 4^k\cdot (2^{2p}+2^{k}-2\cdot 2^{k}\cdot 2^p-2^p+2^k) \\
&=& 4^{i-1} + 2\sum_{k=0}^{i-2} 2^{2k}\cdot (2^{2p}+2^{k}-2\cdot 2^{k}\cdot 2^p-2^p+2^k) \\
&=& 4^{i-1} + 2\sum_{k=0}^{i-2} 2^{2k}\cdot 2^{2p} + 2\sum_{k=0}^{i-2}(2^{k}-2\cdot 2^{k}\cdot 2^p-2^p+2^k) \\
&=& 4^{i-1} + 2\cdot 2^{2p}\sum_{k=0}^{i-2} 2^{2k} + 2\sum_{k=0}^{i-2}2^{3k} - 2^2\cdot 2^p\sum_{k=0}^{i-2}2^{3k} - 2\cdot 2^p\sum_{k=0}^{i-2}2^k + 2\sum_{k=0}^{i-2}2^{2k} \\
&=& 4^{i-1} + 2\cdot 2^{2p}\sum_{k=0}^{i-2} 4^{k} + 2\sum_{k=0}^{i-2}8^{k} - 2^2\cdot 2^p\sum_{k=0}^{i-2}8^{k} - 2\cdot 2^p\sum_{k=0}^{i-2}2^k + 2\sum_{k=0}^{i-2}4^{k} \\
&=& 4^{i-1} + 2\cdot 2^{2p}\frac{4^{i-2}-1}{3} + 2\frac{8^{i-2}-1}{7} - 2^2\cdot 2^p\frac{8^{i-2}-1}{7} - 2\cdot 2^p\frac{2^{i-2}-1}{1} + 2\frac{4^{i-2}-1}{3} \\
&=& 4^{i-1} + 2 i^{2}\frac{4^{i-2}-1}{3} + 2\frac{8^{i-2}-1}{7} - 4 i\frac{8^{i-2}-1}{7} - 2i\frac{2^{i-2}-1}{1} + 2\frac{4^{i-2}-1}{3}
\end{eqnarray*}

\noindent Además, sabemos que si $i\neq j$

$$T_m(i,j) \leq T_m(max(i,j),max(i,j))$$

\noindent Entonces, tenemos que

$$0\leq 2^{max(i,j)} \leq T_m(i,j) $$

\noindent Entonces por definición, concluimos que $$T(n,m) =  \Omega(2^{max(n,m)})$$

\subsection*{Pregunta 4 (Memoizado)}

\noindent Los algoritmos para las funciones group y división permanecen igual. La diferencia con el algoritmo 
recursivo es que almacena los datos ya calculados en una matriz de tamaño $m\ *\ n$ para evitar llamadas que 
calculen datos que ya sabemos. Dado esto y la definición de nuestra recurrencia en la pregunta 2, esta claro que el 
algoritmo va a tener tiempo de ejecución $\Theta(n*m)$ ya que es el tiempo que toma llenar toda la matriz. El algoritmo 
funciona de la siguiente manera:\\
\\\textbf{Recibe:} Dos arreglos de bits, X y Y con la cantidad de pesos que tienen i y j respectivamente.
\\\textbf{Devuelve:} El matching de peso mínimo junto con su peso.

\subsubsection*{Algoritmo: Min-Matching-Memoization}

\begin{algorithmic}[1]
  \TITLE{\textsc{Min-Matching-Memoization}$(X,Y,i,j)$}
    \algcost{\textit{cost}}{\textit{times}}
  \IF{$minMatch[i][j][2]!=\infty$ \algcost{$c_1$}{$1$}} 
    \RETURN $minMatch[i][j][1], \ minMatch[i][j][2]$
      \algcost{$c_2$}{$1$}
  \ENDIF 
  \IF{$i==1$ \AND $j==1$ \algcost{$c_3$}{$1$}} 
    \STATE $minMatch[i][j][1] = [(i,j)]$
      \algcost{$c_4$}{$1$}
    \STATE $minMatch[i][j][2] = \frac{X[i]}{Y[j]}$
      \algcost{$c_5$}{$1$}
    \RETURN $minMatch[i][j][1], \ minMatch[i][j][2]$
      \algcost{$c_6$}{$1$}
  \ENDIF 
  \IF{$i==1$ \AND $j>1$ \algcost{$c_7$}{$1$}} 
    \STATE match = \{\}
      \algcost{$c_8$}{$1$}
    \STATE accumY = 0
      \algcost{$c_9$}{$1$}
  	\FOR{$p=1$ TO $j$ \algcost{$c_{10}$}{$j$}}  
      \STATE accumY += Y[p]
        \algcost{$c_{11}$}{$1$}
      \STATE Match = Match$\ \cup \ \{[i,p]\},$ 
        \algcost{$c_{12}$}{$1$}
    \ENDFOR 
    \STATE $minMatch[i][j][1] = Match$
      \algcost{$c_{13}$}{$1$}
    \STATE $minMatch[i][j][2] = \frac{X[i]}{accumY}$
      \algcost{$c_{14}$}{$1$}
    \RETURN $minMatch[i][j][1], \ minMatch[i][j][2]$
      \algcost{$c_{15}$}{$1$}
  \ENDIF 
  \IF{$i>1$ \AND $j==1$ \algcost{$c_{16}$}{$1$}} 
    \STATE match = \{\}
      \algcost{$c_{17}$}{$1$}
    \STATE accumX = 0
      \algcost{$c_{18}$}{$1$}
  	\FOR{$p=1$ TO $i$ \algcost{$c_{19}$}{$i$}}  
      \STATE accumX += X[p]
        \algcost{$c_{20}$}{$1$}
      \STATE Match = Match$\ \cup \ \{[p,j]\},$ 
        \algcost{$c_{21}$}{$1$}
    \ENDFOR 
    \STATE $minMatch[i][j][1] = Match$
      \algcost{$c_{22}$}{$1$}
    \STATE $minMatch[i][j][2] = \frac{accumX}{Y[j]}$
      \algcost{$c_{23}$}{$1$}
    \RETURN $minMatch[i][j][1], \ minMatch[i][j][2]$
      \algcost{$c_{24}$}{$1$}
  \ENDIF 
  \STATE MatchG, MinG = \textsc{Group}$(X,Y,i,j)$
    \algcost{$i*j$}{$1$}
  \STATE MatchD, MinD = \textsc{Division}$(X,Y,i,j)$
    \algcost{$i*j$}{$1$}  
  \IF{MinG$>$MinD \algcost{$c_{25}$}{$1$}} 
    \RETURN MatchD, MinD
      \algcost{$c_{26}$}{$1$}
  \ENDIF  
  \RETURN MatchG, MinG
    \algcost{$c_{27}$}{$1$}
\end{algorithmic}

\newpage
\subsection*{Pregunta 5 (Programación Dinámica)}

\noindent La primera diferencia que notamos con el algoritmo memoizado es que es iterativo. El algoritmo 
funciona de la siguiente manera: Tenemos una matriz de tamaño $n*m$ la cual va a ser llenada mediante se 
ejecuta el algoritmo. El algoritmo cada vez va añadiendo mas bloques con cada iteración que pasa y decide
que se va a hacer con este nuevo bloque seleccionado. Con esta decisión se va llenando la matriz para que 
la decisión pueda ser usada en las siguientes iteraciones. Por esto es que el algoritmo tiene tiempo de 
ejecución $\Theta(mn^2)$. El código es el siguiente:\\
\\\textbf{Recibe:} Dos arreglos de bits, X y Y.
\\\textbf{Devuelve:} El matching de peso mínimo junto con su peso.

\subsubsection*{Algoritmo: Min-Matching-DP}

\begin{algorithmic}[1]
  \TITLE{\textsc{Min-Matching-DP}$(X,Y,mu)$}
  \algcost{\textit{cost}}{\textit{times}}
  \STATE $accumX = []$
    \algcost{$c_1$}{$1$}
  \STATE $accumY = []$
    \algcost{$c_2$}{$1$}
  \STATE $minMatch = [[[[],math.inf]\ for\ j\ in\ range(len(Y))]\ for\ i\ in\ range(len(X))]$\\
    \algcost{$n*m$}{$1$}
  \STATE $n = len(X)$
    \algcost{$c_3$}{$1$}
  \STATE $m = len(Y)$
    \algcost{$c_4$}{$1$}
  \STATE $accumX.append(X[0])$
    \algcost{$c_5$}{$1$}
  \STATE $accumY.append(Y[0])$
    \algcost{$c_6$}{$1$}
  \FOR{$i=1$ TO n \algcost{$1$}{$n$}}
    \STATE $accumX.append(accumX[i-1]\ +\ X[i])$
      \algcost{$c_7$}{$1$}
  \ENDFOR
  \FOR{$j=1$ TO m \algcost{$1$}{$m$}}
  \STATE $accumY.append(accumY[j-1]\ +\ Y[j])$
    \algcost{$c_7$}{$1$}
  \ENDFOR
  \STATE $minMatch[0][0] = [[(0,0)],X[0]/Y[0]]$
    \algcost{$c_8$}{$1$}
  \FOR{$i = 1$ TO $n$\algcost{$n$}{$1$}}
    \STATE $minMatch[i][0][0] = minMatch[i-1][0][0] + [(i,0)]$
      \algcost{$c_9$}{$1$}
    \STATE $minMatch[i][0][1] = abs(accumX[i] / Y[0] - mu)$
      \algcost{$c_{10}$}{$1$} 
  \ENDFOR
  \FOR{$j = 1$ TO $m$\algcost{$m$}{$1$}}
  \STATE $minMatch[0][j][0] = minMatch[0][j-1][0] + [(0,j)]$
    \algcost{$c_{11}$}{$1$}
  \STATE $minMatch[0][j][1] = abs(X[0] / accumY[j] - mu)$
    \algcost{$c_{12}$}{$1$} 
  \ENDFOR
  \FOR{$i=1$ TO $n$\algcost{$1$}{$n$}}
    \FOR{$j=1$ TO $m$\algcost{$1$}{$m$}}
      \FOR{$k=1$ TO $i$\algcost{$1$}{$i$}}
        \STATE $accum = (accumX[i] - accumX[k]) / Y[j]$
          \algcost{$c_{13}$}{$1$}
        \IF{$minMatch[i][j][1] > abs(accum + pmin - mu))$\algcost{$c_{14}$}{$1$}}
          \STATE $tmatch = [(p,j)\ for\ p\ in\ range(k+1,i+1)]$
            \algcost{$i-k$}{$1$}
          \STATE $minMatch[i][j][0] = match + tmatch$
            \algcost{$c_{15}$}{$1$}
          \STATE $minMatch[i][j][1] = abs(accum + pmin - mu)$
          \algcost{$c_{16}$}{$1$}
        \ENDIF
      \ENDFOR
    \FOR{$k=0$ TO $j$\algcost{$1$}{$j$}}
      \STATE $accum = X[i] / (accumY[j] - accumY[k])$
        \algcost{$c_{17}$}{$1$}
      \STATE $[match, pmin] = minMatch[i-1][k]$
        \algcost{$c_{18}$}{$1$}
      \IF{$minMatch[i][j][1] > abs(accum + pmin - mu))$\algcost{$c_{19}$}{$1$}}
        \STATE $tmatch = [(i,p)\ for\ p\ in\ range(k+1,j+1)]$
          \algcost{$i-k$}{$1$}
        \STATE $minMatch[i][j][0] = match + tmatch$
          \algcost{$c_{20}$}{$1$}
        \STATE $minMatch[i][j][1] = abs(accum + pmin - mu)$
        \algcost{$c_{21}$}{$1$}
      \ENDIF
    \ENDFOR
    \ENDFOR
  \ENDFOR
  \RETURN{$minMatch[n-1][m-1][0], minMatch[n-1][m-1][1]$}
    \algcost{$c_{22}$}{$1$}
  
\end{algorithmic}

\subsection*{Lectura de imágenes}

\noindent En esta parte del proyecto, utilizamos la luminosidad de la imagen para crear la matriz que 
contiene los bloques de bits. Gracias a la libreria imageio de python, podemos cargar una imagen junto 
con los valores RGB de cada pixel. Cuando ya tenemos nuestras dos imágenes, podemos utilizar uno de los 
tres coeficientes de luma disponibles junto con un valor (el umbral) que se pasa a la funcion para ver cuales de los 
pixeles de las imágenes van a volverse un 0 o 1. Los valores mayores o iguales al umbral se vuelven 1 y el resto 0. 
Una vez completada nuestra funcion procedemos a aplicar los algoritmos que veremos en la siguiente sección.

\subsection*{Algoritmos de transformación}

\noindent Para las preguntas de transformaciones de imágenes, estos reciben las dos imágenes como 
parámetros y llaman la funcion luma que devuelve una matriz con los bits correspondientes a cada imagen 
donde están los bloques. Una vez tenemos ambas matrices, se llaman los algoritmos anteriores por cada fila 
para sacar un matching por file de toda la imagen. Como se esta trabajando con imágenes del mismo tamaño, 
el tiempo de ejecución seria el mismo que los algoritmos anteriores multiplicado por la cantidad de 
filas que tienen las imágenes.\\




\subsection*{Link del repositorio}
\noindent \href{https://www.github.com/DiegoELT/ADAProject}{Github link.}


\end{document}
